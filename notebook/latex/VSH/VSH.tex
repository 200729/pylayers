

%% This file was auto-generated by IPython.
%% Conversion from the original notebook file:
%%
\documentclass[11pt,english]{article}

%% This is the automatic preamble used by IPython.  Note that it does *not*
%% include a documentclass declaration, that is added at runtime to the overall
%% document.

\usepackage{amsmath}
\usepackage{amssymb}
\usepackage{graphicx}
\usepackage{ucs}
\usepackage[utf8x]{inputenc}

%fancy verbatim
\usepackage{fancyvrb}
% needed for markdown enumerations to work
\usepackage{enumerate}

% Slightly bigger margins than the latex defaults
\usepackage{geometry}
\geometry{verbose,tmargin=3cm,bmargin=3cm,lmargin=2.5cm,rmargin=2.5cm}

% Define a few colors for use in code, links and cell shading
\usepackage{color}
\definecolor{orange}{cmyk}{0,0.4,0.8,0.2}
\definecolor{darkorange}{rgb}{.71,0.21,0.01}
\definecolor{darkgreen}{rgb}{.12,.54,.11}
\definecolor{myteal}{rgb}{.26, .44, .56}
\definecolor{gray}{gray}{0.45}
\definecolor{lightgray}{gray}{.95}
\definecolor{mediumgray}{gray}{.8}
\definecolor{inputbackground}{rgb}{.95, .95, .85}
\definecolor{outputbackground}{rgb}{.95, .95, .95}
\definecolor{traceback}{rgb}{1, .95, .95}

% new ansi colors
\definecolor{brown}{rgb}{0.54,0.27,0.07}
\definecolor{purple}{rgb}{0.5,0.0,0.5}
\definecolor{darkgray}{gray}{0.25}
\definecolor{lightred}{rgb}{1.0,0.39,0.28}
\definecolor{lightgreen}{rgb}{0.48,0.99,0.0}
\definecolor{lightblue}{rgb}{0.53,0.81,0.92}
\definecolor{lightpurple}{rgb}{0.87,0.63,0.87}
\definecolor{lightcyan}{rgb}{0.5,1.0,0.83}

% Framed environments for code cells (inputs, outputs, errors, ...).  The
% various uses of \unskip (or not) at the end were fine-tuned by hand, so don't
% randomly change them unless you're sure of the effect it will have.
\usepackage{framed}

% remove extraneous vertical space in boxes
\setlength\fboxsep{0pt}

% codecell is the whole input+output set of blocks that a Code cell can
% generate.

% TODO: unfortunately, it seems that using a framed codecell environment breaks
% the ability of the frames inside of it to be broken across pages.  This
% causes at least the problem of having lots of empty space at the bottom of
% pages as new frames are moved to the next page, and if a single frame is too
% long to fit on a page, will completely stop latex from compiling the
% document.  So unless we figure out a solution to this, we'll have to instead
% leave the codecell env. as empty.  I'm keeping the original codecell
% definition here (a thin vertical bar) for reference, in case we find a
% solution to the page break issue.

%% \newenvironment{codecell}{%
%%     \def\FrameCommand{\color{mediumgray} \vrule width 1pt \hspace{5pt}}%
%%    \MakeFramed{\vspace{-0.5em}}}
%%  {\unskip\endMakeFramed}

% For now, make this a no-op...
\newenvironment{codecell}{}

 \newenvironment{codeinput}{%
   \def\FrameCommand{\colorbox{inputbackground}}%
   \MakeFramed{\advance\hsize-\width \FrameRestore}}
 {\unskip\endMakeFramed}

\newenvironment{codeoutput}{%
   \def\FrameCommand{\colorbox{outputbackground}}%
   \vspace{-1.4em}
   \MakeFramed{\advance\hsize-\width \FrameRestore}}
 {\unskip\medskip\endMakeFramed}

\newenvironment{traceback}{%
   \def\FrameCommand{\colorbox{traceback}}%
   \MakeFramed{\advance\hsize-\width \FrameRestore}}
 {\endMakeFramed}

% Use and configure listings package for nicely formatted code
\usepackage{listingsutf8}
\lstset{
  language=python,
  inputencoding=utf8x,
  extendedchars=\true,
  aboveskip=\smallskipamount,
  belowskip=\smallskipamount,
  xleftmargin=2mm,
  breaklines=true,
  basicstyle=\small \ttfamily,
  showstringspaces=false,
  keywordstyle=\color{blue}\bfseries,
  commentstyle=\color{myteal},
  stringstyle=\color{darkgreen},
  identifierstyle=\color{darkorange},
  columns=fullflexible,  % tighter character kerning, like verb
}

% The hyperref package gives us a pdf with properly built
% internal navigation ('pdf bookmarks' for the table of contents,
% internal cross-reference links, web links for URLs, etc.)
\usepackage{hyperref}
\hypersetup{
  breaklinks=true,  % so long urls are correctly broken across lines
  colorlinks=true,
  urlcolor=blue,
  linkcolor=darkorange,
  citecolor=darkgreen,
  }

% hardcode size of all verbatim environments to be a bit smaller
\makeatletter 
\g@addto@macro\@verbatim\small\topsep=0.5em\partopsep=0pt
\makeatother 

% Prevent overflowing lines due to urls and other hard-to-break entities.
\sloppy




\begin{document}


\section{Vector Spherical Harmonics}

In a first step the VSH coefficients are calculated from raw data
describing the full sampled complex radiation pattern over the whole
sphere and for the two orthogonal linear polarizations.

This step can be done once off-line to build an antenna database which
exploits the sparse VSH decomposition.

\[
S_1 : \mathbb{C}^{2 \times N_{\theta} \times N_{\phi} \times N_{f}}
\rightarrow
\mathbb{C}^{4 \times N_{c} }
\]

In a second step field quantities are generated from the VSH
coefficients for a given set of directions corresponding to the DOD/DOA
of the rays coming out the RT engine.

\[
S_2 : \mathbb{C}^{4 \times N_{c} }
\rightarrow
\mathbb{C}^{2 \times N_{r} \times N_{f}}
\]
One of the interest in doing this decomposition comes from the fact that
the number of directions of interest $N_r$ is by far lower than the
number required for the exhaustive sampling of the sphere needed at the
starting point of the synthesis step. In practice, for a given channel,
we do not need to calculate the field in all directions but only for a
limited number of them. The alternative approach would consist in going
through an interpolation procedure into the initial data structure, what
could be either slower if the grid is tight or poorly accurate.

We are dealing with a set of complex quantities that belongs to\\where
$N_{\theta}$,$N_{\phi}$,$N_f$ states respectively for the number of
angle $\theta$ $\phi$ and frequency point. In practice in a ray tracing
tool a direction of \[\mathbb{C}^{2 \times N_{r} \times N_{f}}\] $N_r$
is the number of rays.

\[
% use packages: array
\mathbf{F}(f,\theta,\phi) =
\left[
\begin{array}{l} 
F_{\theta}(f,\theta,\phi)\\ 
F_{\phi}(f,\theta,\phi)
\end{array}
\right]
\]
\subsection{Vector spherical Harmonics function}
The vector spherical harmonics are based on the function
$\bar{V}_l^{(m)}$ and $\bar{W}_l^{(m)}$

\[
\bar{V}_l^{(m)}(x)=(-1)^{n} \frac{\sqrt{1-x^2}}{\sqrt{l(l+1)}}\bar{P}_l^{(m)'}(x)
\]

\[
\bar{V}_l^{(m)}(x)
= \frac{ (-1)^n } {  2\sqrt{l(l+1) } }
  \left( 
  \sqrt{(l+m)(l-m+1)} \bar{P}_l^{(m-1)}(x)
-  \sqrt{(l-m)(l+m+1)} \bar{P}_l^{(m+1)}(x)
  \right)
\]

\[
\bar{W}_l^{(m)}(x)=\frac{(-1)^{n} m}{\sqrt{1-x^2}\sqrt{l(l+1)}}\bar{P}_l^{(m)}(x)
\]

\[
\bar{W}_l^{(m)}(x)=  
\frac{(-1)^{n}}{2x\sqrt{l(l+1)}}
\left[
\sqrt{(l+m)(l-m+1)}
\bar{P}_l^{(m-1)}(x)+
\sqrt{  (l-m) (l+m+1)}   
\bar{P}_l^{(m+1)}(x)
\right]
\]
\subsection{Vector Spherical Harmonics transform step}
This transform is implemented in the spherepack library in Fortran. The
input is the measured complex pattern and the outpur are

4 complex coefficients arrays which are obtained through the following
projection relations.
\[ br_l^{(m)}  =  \alpha_l^{(m)}
\int_{0}^{2\pi}
\int_{-\pi/2}^{-\pi/2}{
( F_{\theta}(\theta,\phi) V_l^{(m)}  \cos{m\phi} - 
  F_{\phi}  (\theta,\phi) W_l^{(m)}  \sin{m\phi})
\cos{\theta} d\theta d\phi} \]

\[
bi_l^{(m)}  =  \alpha_l^{(m)}
\int_{0}^{2\pi}
\int_{-\pi/2}^{-\pi/2}{
( F_{\phi}  (\theta,\phi)  W_l^{(m)} \cos{m\phi} +
  F_{\theta}(\theta,\phi)  V_l^{(m)} \sin{m\phi})
\cos{\theta} d\theta d\phi}
\]

\[
cr_l^{(m)} =  \alpha_l^{(m)}
\int_{0}^{2\pi}
\int_{-\pi/2}^{-\pi/2}{
(  F_{\phi}  (\theta,\phi) V_l^{(m)} \cos{m\phi} +
   F_{\theta}(\theta,\phi) W_l^{(m)} \sin{m\phi})
\cos{\theta} d\theta d\phi}
\]

\[
ci_l^{(m)} =  \alpha_l^{(m)}
\int_{0}^{2\pi} \int_{-\pi/2}^{-\pi/2}{
(  -F_{\theta}(\theta,\phi) W_l^{(m)}  \cos{m\phi} +
    F_{\phi}(\theta,\phi)   V_l^{(m)}  \sin{m\phi})
\cos{\theta} d\theta d\phi}
\]
The normalized Legendre polynomial respectively of order $n$ and $l-1$
are satisfying the following relations

The first step consists in evaluating the vector spherical coefficient
for a set of frequencies over the antenna bandwidth.

The two components of the radiated far field are given by the following
expansion which makes use of 4 coefficients.

The problem here addressed is the one of reconstructing as fast as
possible the radiated field from this set of coefficients.
\[
F_{\phi}(\theta,\phi) = 
\sum_{l=0}^{N}\sum_{m=0}^n  
  W_l^{(m)}(bi_l^{(m)} \cos{m\phi} - br_l^{(m)}\sin{m\phi})  +
  V_l^{(m)}(cr_l^{(m)} \cos{m\phi} + ci_l^{(m)}\sin{m\phi})
\]

\[
F_{\theta}(\theta,\phi) = 
\sum_{l=0}^{N}\sum_{m=0}^n 
    V_l^{(m)} ( br_l^{(m)} \cos{m\phi} + bi_l^{(m)} \sin{m\phi})  +
        W_l^{(m)} (-ci_l^{(m)} \cos{m\phi} + cr_l^{(m)} \sin{m\phi})
\]
In those expressions the expansion functions along the angular parameter
$\theta$ $V_l^{(m)}$ and $W_l^{(m)}$
\[
\sqrt{l(l+1)} V_l^{(m)}(\theta) = 
\frac{dP_l^{(m)}}{d\theta}      =
\frac{1}{2}  P_l^{(m+1)} -(l+m)(l-m+1) P_l^{(m-1)}
\]

\[
\sqrt{l(l+1)} W_l^{(m)}(\theta) =
\frac{m}{\cos\theta}P_l^{(m)}   = 
\frac{1}{2} P_{l-1}^{(m+1)} + (l+m)(l+m-1)P_{l-1}^{(m-1)}
\]

Why $x$ and $\cos(\theta)$ ?

\[
P_l^{(m)} (x)  =  \frac{1}{2^n l!}  (-\cos\theta)^m 
\frac{d^{l+m}}  {dx^{l+m}} (x^2-1)^n
\]

\[
P_{l-1}^{(m)} (x) = \frac{1}{2^{l-1} l!}(-\cos\theta)^m
\frac{d^{l+m-1}} {dx^{l+m-1}} (x^2-1)^{n
-1}
\]
\subsection{Associated Legendre Function}

The SciPy function \emph{lpmn(m,n,z)} is an implementation of the
Associated Legendre functions of the first kind, $P_l^{(m)}(z)$ and its
derivative, $P_l^{'(m)}(z)$ of order $m$ and degree $n$.

It returns two arrays of size $(m+1,l+1)$ containing $P_l^{(m)}(z)$ and
$P_l^{'(m)}(z)$ for all orders from $0 \ldots m$ and degrees from
$0 \ldots n$. The $z$ argument can be complex.
\begin{codecell}
\begin{codeinput}
\begin{lstlisting}
from scipy.special import *
\end{lstlisting}
\end{codeinput}

\end{codecell}
\begin{codecell}
\begin{codeinput}
\begin{lstlisting}
m = 3
n = 3
z = 0.5
lpmn(m,n,z)
\end{lstlisting}
\end{codeinput}
\begin{codeoutput}


\begin{verbatim}
(array([[ 1.        ,  0.5       , -0.125     , -0.4375    ],
       [ 0.        , -0.8660254 , -1.29903811, -0.32475953],
       [ 0.        ,  0.        ,  2.25      ,  5.625     ],
       [ 0.        ,  0.        ,  0.        , -9.74278579]]),
 array([[  0.        ,   1.        ,   1.5       ,   0.375     ],
       [  0.        ,   0.57735027,  -1.73205081,  -6.27868418],
       [  0.        ,   0.        ,  -3.        ,   3.75      ],
       [  0.        ,   0.        ,   0.        ,  19.48557159]]))
\end{verbatim}


\end{codeoutput}
\end{codecell}
\subsubsection{Definition}

In this scipy implementation the Condol-Shortley phase $(-1)^m$ is
already included, thus the expression of the implemented function is
given by :

\[
P_l^{(m)}(x) = \frac{(-1)^m}{2^{n}l!}(1-x^2)^{\frac{m}{2}} \frac{d^{l+m}}{dx^{l+m}}(x^2-1)^n
\]
\subsubsection{Properties}

For associated the two orthogonality relations respectively for
associated Legendre function of different
\subsection{Derivation of Recurrence Formulas on $\bar{P}_l^{(m)}(x)$}

For the Legendre function of the first kind, we have the following
recurrence property

\[
2mx P_l^{(m)}(x)=-\sqrt{1-x^2}
\left[
P_l^{(m+1)}(x)+(l+m)(l-m+1)P_l^{(m-1)}(x)
\right]
\]

Introducing the relation which relates $P_l^{(m)}(x)$ to
$\bar{P}_l^{(m)(x)}$

\[
 P_l^{(m)}(x)= \sqrt{ \frac{2}{2 l+1}  \frac{(l+m)!}{(l-m)!} }  
\bar{P}_l^{(m)}(x)
\]
\[ \frac{2mx}{\sqrt{1-x^2}} \sqrt{\frac{(l+m)!}{(l-m)!} } \bar{P}_l^{(m)}(x) 
=- \left[ \sqrt{  \frac{(l+m+1)!}{(l-m-1)!} }   \bar{P}_l^{(m+1)}(x)
+ (l+m)(l-m+1) \sqrt{  \frac{(l+m-1)!}{(l-m+1)!} }   \bar{P}_l^{(m-1)}(x) \right] \]
\[
 \frac{2mx}{\sqrt{1-x^2}}   
\bar{P}_l^{(m)}(x) =-
\sqrt{   \frac{(l-m)!}{(l+m)!} }\left[
\sqrt{  \frac{(l+m+1)!}{(l-m-1)!} }  
\bar{P}_l^{(m+1)}(x)
+
(l+m)(l-m+1)
\sqrt{  \frac{(l+m-1)!}{(l-m+1)!} }  
\bar{P}_l^{(m-1)}(x)
\right]
\]
\[
\bar{P}_l^{(m)}(x) = -
\frac{\sqrt{1-x^2}}{2mx}
\left[
\sqrt{  (l-m) (l+m+1)}   
\bar{P}_l^{(m+1)}(x)
+
 \sqrt{(l+m)(l-m+1)} 
\bar{P}_l^{(m-1)}(x)
\right]
\]
\subsection{Derivation of Recurrence formulas on
$\frac{d\bar{P}_l^{(m)}}{dx}(x)$}

\[
P_l^{(m)'}(x) = \frac{1}{x^2-1}
\left[
-(l+m)(l-m+1)\sqrt{1-x^2}
P_l^{(m-1)}(x)
-m x 
P_l^{(m)}(x)
\right]
\]

\[
P_l^{(m)'}(x) = \frac{1}{\sqrt{1-x^2}}
\left[
(l+m)(l-m+1)
P_l^{(m-1)}(x)
+ \frac{m x}{\sqrt{1-x^2}} 
P_l^{(m)}(x)
\right]
\]
\[
%P_l^{(m)'}(x) 
\sqrt{   \frac{(l+m)!}{(l-m)!} }  
\bar{P}_l^{(m)'}(x)
= \frac{1}{\sqrt{1-x^2}}
\left[
(l+m)(l-m+1)
\sqrt{  \frac{(l+m-1)!}{(l-m+1)!} }  
\bar{P}_l^{(m-1)}(x)
%P_l^{(m-1)}(x)
+ \frac{m x}{\sqrt{1-x^2}} 
%P_l^{(m)}(x)
\sqrt{   \frac{(l+m)!}{(l-m)!} }  
\bar{P}_l^{(m)}(x)
\right]
\]
\[
%P_l^{(m)'}(x) 
\bar{P}_l^{(m)'}(x)
= \frac{1}{\sqrt{1-x^2}}
\sqrt{   \frac{(l-m)!} {(l+m)!}}  
\left[
(l+m)(l-m+1)
\sqrt{  \frac{(l+m-1)!}{(l-m+1)!} }  
\bar{P}_l^{(m-1)}(x)
%P_l^{(m-1)}(x)
+ \frac{m x}{\sqrt{1-x^2}} 
%P_l^{(m)}(x)
\sqrt{   \frac{(l+m)!}{(l-m)!} }  
\bar{P}_l^{(m)}(x)
\right]
\]
\[  \bar{P}_l^{(m)'}(x) = \frac{1}{\sqrt{1-x^2}}  
\left[
    \sqrt{(l+m)(l-m+1)}
    \bar{P}_l^{(m-1)}(x)
        P_l^{(m-1)}(x)
    + 
    \frac{m x}{\sqrt{1-x^2}} P_l^{(m)}(x)
    \bar{P}_l^{(m)}(x)
\right]
\]
\[ \bar{P}_l^{(m)'}(x) = \frac{1}{2\sqrt{1-x^2}} \left[ \sqrt{(l+m) (l-m+1 )}\bar{P}_l^{(m-1)}(x)-\sqrt{(l-m)(l+m+1)} \bar{P}_l^{(m+1)}(x)  \right]\]

\subsubsection{Evaluation of negative order}

\[
P_l^{(-m)}(x)=(-1)^{m}\frac{(l-m)!}{(l+m)!}P_l^{(m)}(x)
\]
\[
\sqrt{  \frac{(l-m)!}{(l+m)!} }  
\bar{P}_l^{(-m)}(x)
=(-1)^{m}
\frac{(l-m)!}{(l+m)!}
\sqrt{   \frac{(l+m)!}{(l-m)!} }  
\bar{P}_l^{(m)}(x)
\]
\[
\bar{P}_l^{(-m)}(x)
=(-1)^{m}
\frac{(l-m)!}{(l+m)!}
\sqrt{   \frac{(l+m)!}{(l-m)!}  
 \frac{(l+m)!}{(l-m)!} }  
\bar{P}_l^{(m)}(x)
\]
We get the nice and simple relation

\[
\bar{P}_l^{(-m)}(x)
=(-1)^{m}
\bar{P}_l^{(m)}(x)
\]
\subsection{Practical Implementation}

\[
\Re\{F\_{\theta}(\theta,\phi)\} = 
\Re \{ \frac{1}{2} \sum\_{l=0}^{L} 
(br_l^{(0)}\bar{V}_l^{(0)}- j cr_l^{(0)}\bar{W}_l^{(0)})
+ \sum\_{m=1}^{M}\sum\_{l=m}^{N}
  (br_l^{(m)} \bar{V}_l^{m}
-j cr_l^{(m)} \bar{W}_l^{m})
e^{jm\phi}\}
\]
\[
\Im\{F\_{\theta}(\theta,\phi)\} = \Re \{ \frac{1}{2} \sum\_{l=0}^{L} 
(bi_{0,n}\bar{V}_l^{0}- j ci_l^{(0)}\bar{W}_l^{(0)})
+ \sum\_{m=1}^{M}\sum\_{l=m}^{N}(bi_l^{(m)}\bar{V}_l^{m}
-j ci_l^{(m)}\bar{W}_l^{m})
e^{jm\phi}\}
\]
\[
\Re\{F\_{\phi}(\theta,\phi)\} = \Re \{ \frac{1}{2} \sum\_{l=0}^{N} 
(j br_l^{(0)}\bar{W}_l^{(0)}+ cr_l^{(0)}  \bar{V}_l^{(0)})
+ \sum\_{m=1}^{M}\sum\_{l=0}^{N}(j br_l^{(m)} \bar{W}_l^{m}
+cr_l^{(m)}\bar{V}_l^{m})e^{jm\phi})\}
\]
\[
\Im\{F\_{\phi}(\theta,\phi)\} = \Re \{ \frac{1}{2} \sum\_{l=0}^{N} 
(j bi_l^{(0)}\bar{W}_l^{(0)} + ci_l^{(0)}\bar{V}_l^{(0)})
+ \sum\_{m=1}^{M}\sum\_{l=0}^{N}(j bi_l^{(m)}\bar{W}_l^{m}
+ ci_l^{(m)}\bar{V}_l^{m} )
e^{jm\phi}\}
\]
\subsection{The Vector Spherical Harmonics Functions $\bar{V}_l^{(m)}$}

\[
\bar{V}_l^{(m)}(x)
= \frac{ (-1)^n } {  2\sqrt{l(l+1) } }
  \left( 
  \sqrt{(l+m)(l-m+1)} \bar{P}_l^{(m-1)}(x)
-  \sqrt{(l-m)(l+m+1)} \bar{P}_l^{(m+1)}(x)
  \right)
\]
\subsection{Calculation of $\bar{W}_l^{(m)}$}

\[
\bar{W}_l^{(m)}(x)=\frac{(-1)^{n} m}{\sqrt{1-x^2}\sqrt{l(l+1)}}\bar{P}_l^{(m)}(x)
\]
\[
\bar{W}_l^{(m)}(x)=(-1)^{n} \frac{m}{\sqrt{1-x^2}\sqrt{l(l+1)}}
\frac{\sqrt{1-x^2}}{2mx}
\left[
\sqrt{  (l-m) (l+m+1)}   
\bar{P}_l^{(m+1)}(x)
+
 \sqrt{(l+m)(l-m+1)} 
\bar{P}_l^{(m-1)}(x)
\right]
\]
\[
\bar{W}_l^{(m)}(x)= \frac{(-1)^{n}}{2x\sqrt{l(l+1)}}
\left[
\sqrt{  (l-m) (l+m+1)}   
\bar{P}_l^{(m+1)}(x)
+
 \sqrt{(l+m)(l-m+1)} 
\bar{P}_l^{(m-1)}(x)
\right]
\]
\[
\bar{W}_l^{(m)}(\theta)
= \frac{ (-1)^n } {  2 \cos \theta \sqrt{l(l+1) } }
  \left[
  \sqrt{(l-m)(l+m+1)} \bar{P}_l^{(m+1)}(\cos \theta) +
  \sqrt{(l+m)(l-m+1)} \bar{P}_l^{(m-1)}(\cos \theta)
  \right]
\]
There is a singularity of $\bar{W}_l^{(m)}(\theta)$ in $\theta=\pi/2$

\[
\bar{W}_l^{(m)}(\theta)
= \frac{ (-1)^n } {  2 \cos \theta \sqrt{l(l+1) } }
  \left[
  \sqrt{(l-m)(l+m+1)} \bar{P}_l^{(m+1)}(\cos \theta) +
  \sqrt{(l+m)(l-m+1)} \bar{P}_l^{(m-1)}(\cos \theta)
  \right]
\]
\subsection{Norm of $V_l^{(m)}(x)$}

Those basis functions are not unitary
\[
 \int_{-1}^{1} 
 \bar{V}_l^{2(m)}(x) dx = 
\frac{1 } {  4 l(l+1)  }
 \int_{-1}^{1} 
  \left( 
  \sqrt{(l+m)(l-m+1)} \bar{P}_l^{(m-1)}(x)
- \sqrt{(l-m)(l+m+1)} \bar{P}_l^{(m+1)}(x)
  \right)^2
 dx
\]
\[
 \int_{-1}^{1} 
 \bar{V}_l^{2(m)}(x) dx = 
\frac{1 } {  4 l(l+1)  }
 \int_{-1}^{1} 
 ( 
  (l+m)(l-m+1) \bar{P}_l^{2(m-1)}(x)
+ (l-m)(l+m+1) \bar{P}_l^{2(m+1)}(x)\\
- 2\sqrt{(l+m)(l-m+1)(l-m)(l+m+1)}
 \bar{P}_l^{(m-1)}(x)
 \bar{P}_l^{(m+1)}(x)
 )
 dx
\]
\[
 \int_{-1}^{1} 
 \bar{V}\_l^{2(m)}(x) dx = 
\frac{1 } {  4 l(l+1)  }
 \left( (l+m)(l-m+1) 
+ (l-m)(l+m+1)\\
- 2\sqrt{(l+m)(l-m+1)(l-m)(l+m+1)}
\int_{-1}^{1}  
\bar{P}\_l^{(m-1)}(x)
 \bar{P}\_l^{(m+1)}(x) dx
\right)
\]
\[
 \int_{-1}^{1} 
 \bar{V}\_l^{2(m)}(x) dx = 
\frac{1 } {  4 l(l+1)  }
 \left( (l^2-m^2+l+m)
+  (l^2-m^2+l-m)\\
- 2\sqrt{(l^2-m^2+l+m)(l^2-m^2+l-m)}
\int_{-1}^{1}  
\bar{P}\_l^{(m-1)}(x)
 \bar{P}\_l^{(m+1)}(x) dx
\right )
\]
\[
 \int_{-1}^{1} 
 \bar{V}_l^{2(m)}(x) dx = 
\frac{l^2-m^2+l } {  2 l(l+1)  }
 \left( 1 
- \sqrt{1-\frac{m^2}{(l^2-m^2+l)^2}}
\int_{-1}^{1}  
\bar{P}\_l^{(m-1)}(x)
 \bar{P}\_l^{(m+1)}(x) dx
 \right)
\]
\subsection{Norm of $\bar{W}_l^{(m)}(x)$}

\[
 \int_{-1}^{1} 
 \bar{W}_l^{2(m)}(x) dx = 
\frac{ l^2-m^2+l } { l(l+1) } 
\left(
  1
- \sqrt{1-\frac{m^2}{(l^2-m^2+l)^2}}
  \int_{-1}^{1} \bar{P}_l^{(m-1)}(x)\bar{P}_l^{(m+1)}(x) dx
  \right)
\]



\end{document}

